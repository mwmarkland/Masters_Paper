\section{Introduction}
\label{introduction}
The goal of Larch/C++ \cite{Leavens96c} is to give the programmer a
formal specification language that is expressive and useful in
practice. The work described in this paper focuses on the addition of
type checking capabilities to the Larch/C++ Checker. In
this context a
\emph{type} is a set of values that exhibit uniform behavior under
a set of associated operations \cite{Watt90}. \emph{Type checking}
defines a process by which a set of formal rules that describe the
type system are applied to operations and statements in a given
specification to decide whether the operations and statements are type
consistent (the terms \emph{sort} and \emph{sort checking} will also
be used interchangeably for type and type checking). This functionality allows programmers and specifiers to gain useful
information as to whether the design of a program is sensible and type
consistent before its actual implementation.  

Work has been done in the following areas:

\begin{itemize}
\item The creation of a formal description of the Larch/C++ type system.

\item The coding of basic functionality to support the 
implementation of the Larch/C++ type system. This work has many
smaller pieces including:

\begin{itemize}
\item Support for User Interaction 
\begin{itemize}
\item Improved usability and control over the Larch/C++ Checker via the
creation of a number of command line arguments.
\end{itemize}

\item Support for Larch Shared Language Constructs
\begin{itemize}
\item Modification of the LSL Checker to follow Unix conventions and
use less memory.
\item Development of an interface between the existing Larch/C++ Checker and
the LSL Checker.
\end{itemize}

\item Support for the Evolving C++ Language Standard
\begin{itemize}
\item Support for Draft ANSI Standard C++ \cite{C++Apr95}
language constructs.
\item Support for translation of C++ declarations into Larch/C++
sorts.
\end{itemize}
\end{itemize}
\end{itemize}

The paper begins with a background section that introducing formal
methods, specification, and the Larch/C++ behavioral interface
specification language. Following the general introduction, the type
systems of the Larch Shared Language and C++ are described
briefly. These descriptions motivate a discussion of the basic
functionality of the Larch/C++ type system. This functionality is then
formalized in a description of the sort rules for Larch/C++. Finally
the details of the implementation of the functionality mentioned above
are presented.

\section{Background}
\label{background}
\subsection{Formal Methods}

The information in this section is based upon Wing's
paper \cite{Wing90a}. 

Formal methods define processes that are used for software
development. Built upon a mathematical basis, these processes are
designed to reveal ambiguities, incompleteness, and inconsistencies in
software as it is developed. A formal method will
typically define the specific vocabulary and steps involved in
designing a piece of software.

Formal specifications may be a part of a specific formal
method. \emph{Formal specification} describes a process by which an
abstraction of a problem may be defined and expressed. Usually the
abstraction is expressed in some language specifically
designed for the purpose. Once clearly expressed, the abstraction may serve
as documentation of the problem, a means to communicate the problem
clearly, and/or as a contract defining the problem. Specifications and
their languages are based upon mathematical properties making
them more precise and concise than informal specifications based upon
natural languages. The rigorous definition and mathematical basis
behind formal specification languages also makes it easier to apply machine
analysis and manipulations to them than to specifications in an
informal language.

Larch/C++ is a behavioral interface specification language. A \emph{behavioral}
specification language is used to define an abstraction for a system
based upon the behavior of that system under certain conditions. In
other words, it describes a system's behavior as observed from the
outside. Larch/C++ describes behavior using a
\emph{model-based} approach; a user builds an abstract model of the system
which describes its behavior. The abstract model then becomes a way of
expressing the real world in a manner that can be controlled and
reasoned about. The basic pieces of the model are the interfaces which
exist between the pieces of the system. After a user has described the
behavior in terms of the interfaces, solutions to the problem may then
be designed based upon the formal contract defined by the model.

%One common type of formal specification is
%model-based. \emph{Model-based} specification involves the creation
%of an abstract model of the problem that is independent of any actual
%implementation to serve as the specification. The abstract model
%becomes a way of expressing the real world in a manner that can be
%controlled and reasoned about. Once the model has been developed, a
%solution based upon this model can be formalized. Since the solution
%has a rigorous, formal, and mathematical basis, it may be reasoned
%about to decide if it meets the requirements. Once the solution has
%been deemed correct, the specification becomes a formal contract
%describing how the implementation should behave under certain
%conditions.

%There are many different layers at which one may specify a
%problem. Larch/C++ is based upon the idea of behavioral
%specification. \emph{Behavioral specifications} describe the
%constraints on the observable behavior of the items specified. In this
%way, they form a contract between an implementor and a user of the
%implementation.


\subsection{Introduction to Larch/C++}
\label{lcppintro}
Larch/C++ \cite{Leavens96c} is a model-based, formal specification language tailored for
the specification of the behavior of C++ program modules or
application program interfaces (API's). Larch/C++ is not designed to
specify the 
behavior of an entire program; instead it allows for the precise,
unambiguous documentation of the behavior C++ program modules
(functions, classes, etc.). Larch/C++ adds syntax to C++ to allow the
specification of complex C++ structures, the inheritance of
specifications, and the clear specification of the interface to a
class.

Larch/C++ is a two-tiered specification language. Specifications
consist of Larch Shared Language (LSL) \cite{Guttag-Horning93} traits
which describe the abstract models, and interface specifications which
formalize the behavioral contracts. This two-tiered approach allows
for the clear separation of the definition of the abstract model and its
vocabulary from the actual details of the programming language modeled
by the interface language. One reason for the separation is so that
the abstract models may be written so that they can be reused in other
specifications. If the abstract models were written in a language
closer to the actual implementation, it would be more difficult to
reuse the same model in a specification for an implementation written
in a different language.

\subsubsection{The Larch Shared Language}
\label{lslintro}
The Larch Shared Language \cite{Guttag-Horning93} allows an user to
supply basic semantic information, and a specialized vocabulary for
describing abstract values. The basic unit of LSL is the
\emph{trait}. Traits contain information on \emph{sorts}, which are
like types in a programming language, and \emph{operators} which define
various operations upon these sorts. Figure~\ref{CounterTrait}
illustrates the LSL portion of a specification for a simple counter
\cite{Leavens96c}. This particular trait illustrates four common
parts that are in many traits. The
\reserved{includes} statement allows for a trait to build upon and
reuse previously written traits. All of the information from the
traits listed in this statement are available in the following
sections of the current trait. Items from the included traits may be
renamed syntactically to make the new trait more readable. This
provides a shortcut for users, allowing them to reuse previous work.
The \reserved{introduces} section defines the abstract model's
operations. In this case, a Counter may be created via
\reserved{newCounter} or \reserved{inc}, and have its
value reported via \reserved{value}. The
\reserved{asserts} section supplies meaning to these
operations by logically illustrating how the abstract values are
manipulated by the operators. For now, note that a trait defines a
model consisting of a set of abstract values and a set of operations
upon those values. Examples of possible abstract values for the trait
in Figure~\ref{CounterTrait} are \reserved{newCounter}, representing a
brand new counter, and \reserved{inc(newCounter)} representing a
counter that has been incremented once. These values are independent
of any implementation of Counter.

\begin{BFIGURE}
\begin{verbatim}
 @(#)$Id: CounterTrait.lsl,v 1.3 1994/12/09 02:48:06 leavens Exp $
CounterTrait: trait
  includes Natural(Nat), NoContainedObjects(Counter)
  introduces
    newCounter: -> Counter
    inc: Counter -> Counter
    value: Counter -> Nat
    Limit: -> Nat
  asserts
    Counter generated by newCounter, inc
    Counter partitioned by value
    \forall c: Counter
      value(newCounter) == 0;
      value(inc(c)) == value(c) + 1;
      0 < Limit;
\end{verbatim}

\caption{CounterTrait.lsl}
\label{CounterTrait}
\end{BFIGURE}


\subsubsection{The Larch/C++ Specification Language}
\label{lcppbisl}
Figure~\ref{CounterSpec} contains the Larch/C++ specification
for a class that implements the counter modeled by the previous trait
\cite{Leavens96c}. Larch/C++ specifications consist of a C++ header
file which contains additional annotations set off in specially marked
C++ comments. The use of the comment delimiters, \reserved{//@} and
\reserved{/*@ $\dots$ @*/}, allows the annotations to be easily embedded
directly into new or existing C++ source code. Larch/C++ annotations contain various keywords. In this example, the first section is a
\emph{uses-clause}, which serves a function similar to the \reserved{\#include}
directive in C++; it tells the Larch/C++ Checker the traits that will
be used by this specification. Following that, the next annotations
define an \reserved{invariant} and a
\reserved{constraint}. The
\reserved{invariant} describes a condition for the C++ class that must
always be true. The \reserved{constraint} describes any limits that
this class must adhere to. 

Most individual function specifications
consist of at least three pieces: a requires clause, a modifies
clause, and an ensures clause. The \emph{requires clause} states the
conditions that must be met before an individual function may be
called. If these conditions are not met, there is no guarantee that
the function will run correctly. The \emph{modifies} clause lists all
objects whose state may be initialized or modified by the execution of
this function. Only objects listed
in this clause may change state; thus the modifies clause acts as a
frame axiom. The \emph{ensures clause} serves to state the expected
results of the function provided the conditions specified in the
requires clause were met.

A closer look at the \reserved{increment} function specification illustrates
these sections.
 

\begin{verbatim}
  virtual void increment();
  //@ behavior {
  //@   requires value(self^) < Limit;
  //@   modifies self;
  //@   ensures  self' = inc(self^);
  //@ }
\end{verbatim}

%\begin{verbatim}
%  virtual void increment();
%  //@ behavior {
%  //@   requires value^ < Limit;
%  //@   modifies value;
%  //@   ensures  value' = value^ + 1;
%  //@ }
%\end{verbatim}

\noindent The first line is the C++ prototype for the function. The
\reserved{behavior} keyword announces the beginning of the body of the
specification. The requires clause states that for this function to be
called the value of the counter must be less than the value of
\reserved{Limit}. If it is not, then the behavior of the function is
not specified. If the requires clause is met and function is called,
the modifies clause states that the only possible item that can change
is the counter itself. The ensures clause then states that, if the
previous conditions in the requires clause are met, the value of the counter after the
invocation of this function will be the result of incrementing the
counter. 
 
\begin{BFIGURE}
\begin{verbatim}
// @(#)$Id: Counter.lh,v 1.7 1997/01/12 22:27:38 leavens Exp $
// See J.P. Lejacq's paper in SIGPLAN 26(10), Oct, 1991

//@ uses CounterTrait;

class Counter {
public:
  //@ invariant _value(self\any) <= Limit;
  //@ constraint value(self\pre) <= value(self\post);

  Counter();
  //@ behavior {
  //@   modifies self;
  //@   ensures  self' = newCounter;
  //@ }

  virtual int cnt_value() const;
  //@ behavior {
  //@   ensures result = value(self^);
  //@ }

  virtual void increment();
  //@ behavior {
  //@   requires value(self^) < Limit;
  //@   modifies self;
  //@   ensures  self' = inc(self^);
  //@ }

  virtual void reset();
  //@ behavior {
  //@   modifies self;
  //@   ensures  value(self') = 0;
  //@ }
};

\end{verbatim}
\caption{Counter.lh}
\label{CounterSpec}
\end{BFIGURE}
