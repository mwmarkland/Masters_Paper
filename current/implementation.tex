\section{Implementation of the System} 
As mentioned in the
Introduction, the development process for type checking in the
Larch/C++ Checker has occurred in many pieces. This section of the
paper begins with a short description of the Checker's source files
and the tools used in development. Following that, the modifications to
the Checker to support the C++ \reserved{namespace} construction are
described. A description of the modifications to the Larch/C++ Checker
and the LSL Checker to allow for the support of LSL constructs in
Larch/C++ continues the section. Then a description of how
the system converts C++ declarations to LSL sorts is given. The
concluding section discusses the lookup algorithm and preliminary type
checking results. Note that the source code presented in the following
sections has been stripped of any comments or specifications in order to save
space. Please refer to the actual implementation for more details.

\subsection{Overview of the System}
The task of creating tools for a language as complex as Larch/C++ is
difficult. Any implementation of the tools will also be complex.
Larch/C++'s implementation can be broken down into the following major pieces:
\begin{itemize}

\item The lexer and parser for Larch/C++
\item The lexer and parser for LSL operation signatures
\item The temporary file support system
\item The LSL Checker
\item The symbol table system

\end{itemize}

\noindent Note that the LSL Checker is included here because it required
modification to work within the system, and because it is an important 
part of the system as a whole.
 
\begin{BFIGURE}
\begin{center}
\setlength{\unitlength}{0.00083300in}%
%
\begingroup\makeatletter\ifx\SetFigFont\undefined
% extract first six characters in \fmtname
\def\x#1#2#3#4#5#6#7\relax{\def\x{#1#2#3#4#5#6}}%
\expandafter\x\fmtname xxxxxx\relax \def\y{splain}%
\ifx\x\y   % LaTeX or SliTeX?
\gdef\SetFigFont#1#2#3{%
  \ifnum #1<17\tiny\else \ifnum #1<20\small\else
  \ifnum #1<24\normalsize\else \ifnum #1<29\large\else
  \ifnum #1<34\Large\else \ifnum #1<41\LARGE\else
     \huge\fi\fi\fi\fi\fi\fi
  \csname #3\endcsname}%
\else
\gdef\SetFigFont#1#2#3{\begingroup
  \count@#1\relax \ifnum 25<\count@\count@25\fi
  \def\x{\endgroup\@setsize\SetFigFont{#2pt}}%
  \expandafter\x
    \csname \romannumeral\the\count@ pt\expandafter\endcsname
    \csname @\romannumeral\the\count@ pt\endcsname
  \csname #3\endcsname}%
\fi
\fi\endgroup
\begin{picture}(6107,4534)(364,-4300)
\thicklines
\special{ps: gsave 0 0 0 setrgbcolor}\put(2932,-3954){\framebox(3145,3931){}}
\special{ps: gsave 0 0 0 setrgbcolor}\put(3456,-3692){\framebox(787,524){}}
\special{ps: gsave 0 0 0 setrgbcolor}\put(5029,-3692){\framebox(786,524){}}
\special{ps: gsave 0 0 0 setrgbcolor}\put(573,-2578){\framebox(1179,524){}}
\special{ps: gsave 1 0 0 setrgbcolor}\put(4767,-1858){\framebox(1048,524){}}
\special{ps: gsave 0 0 0 setrgbcolor}\put(5553,-3168){\vector( 0, 1){1310}}
\special{ps: grestore}\special{ps: gsave 0 0 0 setrgbcolor}\put(5553,-1334){\vector( 0, 1){263}}
\special{ps: grestore}\special{ps: gsave 0 0 0 setrgbcolor}\put(5029,-1071){\vector( 0,-1){263}}
\special{ps: grestore}\special{ps: gsave 0 0 0 setrgbcolor}\put(5029,-1858){\vector( 0,-1){262}}
\special{ps: grestore}\special{ps: gsave 0 0 .69 setrgbcolor}\put(3980,-2906){\framebox(1311,786){}}
\special{ps: gsave 0 0 0 setrgbcolor}\put(5094,-2906){\vector( 0,-1){262}}
\special{ps: grestore}\put(5815,-678){\vector( 1, 0){525}}
\special{ps: gsave 0 0 0 setrgbcolor}\put(5815,-3299){\vector( 4, 3){524.640}}
\special{ps: grestore}\special{ps: gsave 0 0 0 setrgbcolor}\put(5815,-3496){\vector( 1,-1){491.500}}
\special{ps: grestore}\special{ps: gsave 0 0 0 setrgbcolor}\put(5815,-3430){\vector( 1, 0){525}}
\special{ps: grestore}\special{ps: gsave 1 0 1 setrgbcolor}\put(4505,-1071){\framebox(1310,786){}}
\special{ps: gsave 0 0 0 setrgbcolor}\put(638,-1792){\framebox(1049,524){}}
\special{ps: gsave 0 0 0 setrgbcolor}\put(1752,-2382){\line( 1, 0){656}}
\put(2408,-2382){\line( 0,-1){1048}}
\put(2408,-3430){\vector( 1, 0){1048}}
\special{ps: grestore}\special{ps: gsave .56 0 .56 setrgbcolor}\put(4505,-3299){\vector( 1, 0){524}}
\put(4505,-3299){\line( 0, 1){262}}
\put(4505,-3037){\line(-1, 0){787}}
\put(3718,-3037){\line( 0, 1){2097}}
\put(3718,-940){\vector( 1, 0){787}}
\special{ps: grestore}\special{ps: gsave .56 0 .56 setrgbcolor}\put(1687,-1530){\line( 1, 0){721}}
\put(2408,-1530){\line( 0, 1){917}}
\put(2408,-613){\vector( 1, 0){2097}}
\special{ps: grestore}\special{ps: gsave 0 0 0 setrgbcolor}\put(4243,-3430){\vector(-1, 0){  0}}
\put(4243,-3430){\vector( 1, 0){786}}
\special{ps: grestore}\put(3001,-2986){\framebox(600,525){}}
\put(3601,-2686){\vector(-1, 0){  0}}
\put(3601,-2686){\vector( 1, 0){375}}
\put(3226,-3286){\vector( 1, 0){225}}
\put(3226,-3286){\vector( 0, 1){300}}
\special{ps: gsave 0 0 0 setrgbcolor}\put(376,-3103){\framebox(1573,2097){}}
\put(3226,-3286){\line( 0,-1){600}}
\put(3226,-3886){\line( 1, 0){2175}}
\put(5401,-3886){\vector( 0, 1){225}}
\put(3076,-2686){\makebox(0,0)[lb]{\smash{\SetFigFont{9}{10.8}{sf}SymTab}}}
\put(2997,108){\makebox(0,0)[lb]{\smash{\SetFigFont{10}{12.0}{sf}\special{ps: gsave 0 0 0 setrgbcolor}lcppparse\special{ps: grestore}}}}
\put(704,-2251){\makebox(0,0)[lb]{\smash{\SetFigFont{10}{12.0}{sf}\special{ps: gsave 0 0 0 setrgbcolor}Larch/C++\special{ps: grestore}}}}
\put(704,-2447){\makebox(0,0)[lb]{\smash{\SetFigFont{10}{12.0}{sf}\special{ps: gsave 0 0 0 setrgbcolor}Statements\special{ps: grestore}}}}
\put(4898,-1661){\makebox(0,0)[lb]{\smash{\SetFigFont{10}{12.0}{sf}\special{ps: gsave 1 0 0 setrgbcolor}TmpFiles\special{ps: grestore}}}}
\put(4046,-2316){\makebox(0,0)[lb]{\smash{\SetFigFont{10}{12.0}{sf}\special{ps: gsave 0 0 .69 setrgbcolor}operation\special{ps: grestore}}}}
\put(4046,-2513){\makebox(0,0)[lb]{\smash{\SetFigFont{10}{12.0}{sf}\special{ps: gsave 0 0 .69 setrgbcolor}signature\special{ps: grestore}}}}
\put(4046,-2709){\makebox(0,0)[lb]{\smash{\SetFigFont{10}{12.0}{sf}\special{ps: gsave 0 0 .69 setrgbcolor}parser\special{ps: grestore}}}}
\put(6405,-3365){\makebox(0,0)[lb]{\smash{\SetFigFont{10}{12.0}{sf}\special{ps: gsave 0 0 0 setrgbcolor}syntax\special{ps: grestore}}}}
\put(6405,-3561){\makebox(0,0)[lb]{\smash{\SetFigFont{10}{12.0}{sf}\special{ps: gsave 0 0 0 setrgbcolor}errors\special{ps: grestore}}}}
\put(6405,-744){\makebox(0,0)[lb]{\smash{\SetFigFont{10}{12.0}{sf}\special{ps: gsave .82 0 0 setrgbcolor}errors\special{ps: grestore}}}}
\put(6405,-2775){\makebox(0,0)[lb]{\smash{\SetFigFont{10}{12.0}{sf}\special{ps: gsave .82 0 0 setrgbcolor}semantic\special{ps: grestore}}}}
\put(6405,-2972){\makebox(0,0)[lb]{\smash{\SetFigFont{10}{12.0}{sf}\special{ps: gsave .82 0 0 setrgbcolor}errors\special{ps: grestore}}}}
\put(6471,-3889){\makebox(0,0)[lb]{\smash{\SetFigFont{10}{12.0}{sf}\special{ps: gsave .82 0 0 setrgbcolor}trait\special{ps: grestore}}}}
\put(6471,-4085){\makebox(0,0)[lb]{\smash{\SetFigFont{10}{12.0}{sf}\special{ps: gsave .82 0 0 setrgbcolor}inclusion\special{ps: grestore}}}}
\put(6471,-4282){\makebox(0,0)[lb]{\smash{\SetFigFont{10}{12.0}{sf}\special{ps: gsave .82 0 0 setrgbcolor}errors\special{ps: grestore}}}}
\put(4570,-744){\makebox(0,0)[lb]{\smash{\SetFigFont{10}{12.0}{sf}\special{ps: gsave 1 0 1 setrgbcolor}LSL Checker\special{ps: grestore}}}}
\put(376,-875){\makebox(0,0)[lb]{\smash{\SetFigFont{10}{12.0}{rm}\special{ps: gsave 0 0 0 setrgbcolor}Specification\special{ps: grestore}}}}
\put(376,-678){\makebox(0,0)[lb]{\smash{\SetFigFont{10}{12.0}{sf}\special{ps: gsave 0 0 0 setrgbcolor}Larch/C++\special{ps: grestore}}}}
\put(704,-1596){\makebox(0,0)[lb]{\smash{\SetFigFont{10}{12.0}{sf}\special{ps: gsave 0 0 0 setrgbcolor}LSL Traits\special{ps: grestore}}}}
\put(5160,-3496){\makebox(0,0)[lb]{\smash{\SetFigFont{10}{12.0}{sf}\special{ps: gsave 0 0 0 setrgbcolor}parser\special{ps: grestore}}}}
\put(3822,-1389){\makebox(0,0)[lb]{\smash{\SetFigFont{10}{12.0}{sf}\special{ps: gsave .56 0 .56 setrgbcolor}control\special{ps: grestore}}}}
\put(3822,-1585){\makebox(0,0)[lb]{\smash{\SetFigFont{10}{12.0}{sf}\special{ps: gsave .56 0 .56 setrgbcolor}signals\special{ps: grestore}}}}
\put(3653,-3496){\makebox(0,0)[lb]{\smash{\SetFigFont{10}{12.0}{sf}\special{ps: gsave 0 0 0 setrgbcolor}lexer\special{ps: grestore}}}}
\end{picture}

\end{center}
\caption{The structure of the Larch/C++ Checker}
\label{sysover2}
\end{BFIGURE}

\noindent Figure~\ref{sysover2} shows a pictorial representation of the
structure of the system.

The majority of the Larch/C++ Checker's implementation lies within the
lexer, the parser, and the support code. The lexer and parser are
built using an attribute grammar system called Ox~\cite{Ox}. The
Checker does not take full advantage of Ox, but the
syntax for the creation and access of attributes for rules in Ox can
be clearer than the equivalent expressions in \reserved{yacc}. Ox takes a
decorated grammar as input and generates output in the form of
\reserved{.l} and \reserved{.y} files to be sent through \reserved{lex} and
\reserved{yacc}. 

The support files for the lexer and parser include all of the code to
generate the various data structures used within the Checker. The code
has been developed using C++. Names of the source files and classes
attempt to clearly state what they are used for. The file names can be
confusing because they are limited to 8.3 format for compatibility
with MS-DOS (Originally, MS-DOS allowed only filenames which had eight 
characters followed by a three character extension). An example of how the files and classes are named is the
case of type specifiers. Type specifiers are a part of the C++
declaration syntax. The support code for type specifiers in the
Larch/C++ Checker is contained in files named \reserved{TypeScfr.*}
and the class implemented is actually named \reserved{TypeSpecifier}. It is
important to note that some header files are broken into four
pieces, the \reserved{.pre} file, the \reserved{.h} file, the
\reserved{.pri} file, and the \reserved{.bse} file. Not all headers
have all of these files, but most have at least the \reserved{.h} and
\reserved{.pri}. The \reserved{.pre} file is used to hold any
private declarations or \reserved{\#include} directives for the
\reserved{.h} file that might occur before a class definition. The
\reserved{.h} file contains the public portion of the class definition. The
\reserved{.pri} file contains protected or private members. The
\reserved{.bse} file is used for private or protected inheritance.

The common code portion of the system contains code that is not
specific to the Larch/C++ Checker. This includes implementations of
dynamic strings, debugging code, interfaces to the environment, and
other functions. These classes and functions are used throughout the
whole of the system.

Finally, not mentioned in the above list but still very important to
the Larch/C++ Checker is the Boehm-Demers-Weiser Conservative Garbage
Collector~\cite{Boehm-Weiser88}. All reclamation of allocated storage
is handled by the garbage collector. This is important because the
code makes heavy use of pointers and dynamic allocation of objects.

\subsection{Support for C++ \reserved{namespace}}
\label{namspcimp}
The Draft C++ Standard~\cite{C++Apr95} contains definitions for many
new language constructs. One of these is \reserved{namespace}. The
\reserved{namespace} construct was added to C++ to allow for 
an additional level of scoping for names. Before namespaces, the
possibility existed for vendors to supply libraries of code that would
have name conflicts. For example, vendor A ships a linked-list library
with the operation \reserved{Delete}. It was possible that this
\reserved{Delete} could name conflict with some other piece of
code. The only solution was either to not use the library or to wrap
the library in another class to isolate the name. Namespaces allow for
declarations to be wrapped in another name without the need for
additional classes. These wrapped names can then be made available via
a \reserved{using} declaration.
\begin{BFIGURE}
\begin{verbatim}
namespace foo{
  int i;
  int j;
}

void inc(){
  using namespace foo;
}

void dec(){
  using foo::i;
}
\end{verbatim}
\caption{The C++ \reserved{namespace} construction}
\label{namespcex}
\end{BFIGURE}

Figure~\ref{namespcex} shows an example of the use of
the \reserved{namespace} construction.The \reserved{namespace} declaration creates a new \reserved{namespace} named
\reserved{foo} which contains the declarations for \reserved{i} and
\reserved{j}. These variables will not be visible at any scope level
(see Section~\ref{cppts} for a review of the C++ scoping system)
unless there is a \reserved{using} directive or a
\reserved{using} declaration involving them. Within the function
\reserved{inc()} there is an example of a \reserved{using}
directive. The declaration 
\begin{verbatim} 
using namespace foo; 
\end{verbatim}
informs the system that all the declarations contained within
\reserved{namespace foo} should become visible at the current
scope. Similarly the \reserved{using} declaration: 
\begin{verbatim} 
uses foo::i;
\end{verbatim} 
informs the system that the declaration for \reserved{i} contained in
\reserved{namespace foo} should be visible at the current
scope. Another form of the \reserved{namespace} declaration, the
\emph{unnamed namespace}, has a similar syntax, except a name is not
supplied for the namespace. The semantics of unnamed namespaces
are a little different, however. An unnamed \reserved{namespace} has the effect
of declaring a namespace followed by an immediate \reserved{using}
directive. Thus, the names become visible immediately.

\begin{BFIGURE}
\begin{center}
\setlength{\unitlength}{0.00083300in}%
%
\begingroup\makeatletter\ifx\SetFigFont\undefined
% extract first six characters in \fmtname
\def\x#1#2#3#4#5#6#7\relax{\def\x{#1#2#3#4#5#6}}%
\expandafter\x\fmtname xxxxxx\relax \def\y{splain}%
\ifx\x\y   % LaTeX or SliTeX?
\gdef\SetFigFont#1#2#3{%
  \ifnum #1<17\tiny\else \ifnum #1<20\small\else
  \ifnum #1<24\normalsize\else \ifnum #1<29\large\else
  \ifnum #1<34\Large\else \ifnum #1<41\LARGE\else
     \huge\fi\fi\fi\fi\fi\fi
  \csname #3\endcsname}%
\else
\gdef\SetFigFont#1#2#3{\begingroup
  \count@#1\relax \ifnum 25<\count@\count@25\fi
  \def\x{\endgroup\@setsize\SetFigFont{#2pt}}%
  \expandafter\x
    \csname \romannumeral\the\count@ pt\expandafter\endcsname
    \csname @\romannumeral\the\count@ pt\endcsname
  \csname #3\endcsname}%
\fi
\fi\endgroup
\begin{picture}(4512,2724)(139,-2473)
\thicklines
\put(751,-2161){\makebox(6.6667,10.0000){\SetFigFont{10}{12}{rm}.}}
\put(751,-2161){\vector( 1, 0){750}}
\put(151,-1861){\framebox(1200,300){}}
\put(226,-61){\framebox(1200,300){}}
\put(676,-61){\line( 0,-1){300}}
\put(676,-361){\makebox(6.6667,10.0000){\SetFigFont{10}{12}{rm}.}}
\put(676,-361){\vector( 1, 0){825}}
\put(2701,-661){\framebox(1575,600){}}
\put(4126,-211){\vector( 1, 0){450}}
\put(4126,-511){\vector( 1, 0){450}}
\put(2701,-361){\line( 1, 0){1575}}
\put(1501,-2461){\framebox(900,600){}}
\put(1501,-2161){\line( 1, 0){900}}
\put(2176,-2311){\vector( 1, 0){525}}
\put(1501,-1561){\framebox(900,600){}}
\put(1501,-1261){\line( 1, 0){900}}
\put(2176,-1411){\vector( 1, 0){525}}
\put(1501,-661){\framebox(900,600){}}
\put(1501,-361){\line( 1, 0){900}}
\put(2176,-511){\vector( 1, 0){525}}
\put(2701,-1561){\framebox(1575,600){}}
\put(4126,-1111){\vector( 1, 0){450}}
\put(4126,-1411){\vector( 1, 0){450}}
\put(2701,-1261){\line( 1, 0){1575}}
\put(751,-1861){\line( 0,-1){300}}
\put(2701,-2461){\framebox(1575,600){}}
\put(4651,-2086){\makebox(0,0)[lb]{\smash{\SetFigFont{10}{12.0}{tt}Symbols ...}}}
\put(4126,-2011){\vector( 1, 0){450}}
\put(4126,-2311){\vector( 1, 0){450}}
\put(2701,-2161){\line( 1, 0){1575}}
\put(2326,-2011){\vector( 0, 1){450}}
\put(2326,-1111){\vector( 0, 1){450}}
\put(151,-1786){\makebox(0,0)[lb]{\smash{\SetFigFont{10}{12.0}{tt}currentSymTab}}}
\put(301, 14){\makebox(0,0)[lb]{\smash{\SetFigFont{10}{12.0}{tt}globalSymTab}}}
\put(2776,-286){\makebox(0,0)[lb]{\smash{\SetFigFont{10}{12.0}{tt}nonClassOrEnum}}}
\put(2776,-586){\makebox(0,0)[lb]{\smash{\SetFigFont{10}{12.0}{tt}classOrEnum}}}
\put(4576,-286){\makebox(0,0)[lb]{\smash{\SetFigFont{10}{12.0}{tt}Symbols ...}}}
\put(4576,-586){\makebox(0,0)[lb]{\smash{\SetFigFont{10}{12.0}{tt}Symbols ...}}}
\put(1576,-2086){\makebox(0,0)[lb]{\smash{\SetFigFont{10}{12.0}{tt}globals}}}
\put(1576,-2386){\makebox(0,0)[lb]{\smash{\SetFigFont{10}{12.0}{tt}locals}}}
\put(1576,-1186){\makebox(0,0)[lb]{\smash{\SetFigFont{10}{12.0}{tt}globals}}}
\put(1576,-1486){\makebox(0,0)[lb]{\smash{\SetFigFont{10}{12.0}{tt}locals}}}
\put(1576,-286){\makebox(0,0)[lb]{\smash{\SetFigFont{10}{12.0}{tt}globals}}}
\put(1576,-586){\makebox(0,0)[lb]{\smash{\SetFigFont{10}{12.0}{tt}locals}}}
\put(2776,-1186){\makebox(0,0)[lb]{\smash{\SetFigFont{10}{12.0}{tt}nonClassOrEnum}}}
\put(2776,-1486){\makebox(0,0)[lb]{\smash{\SetFigFont{10}{12.0}{tt}classOrEnum}}}
\put(4651,-1186){\makebox(0,0)[lb]{\smash{\SetFigFont{10}{12.0}{tt}Symbols ...}}}
\put(4651,-1486){\makebox(0,0)[lb]{\smash{\SetFigFont{10}{12.0}{tt}Symbols ...}}}
\put(2776,-2086){\makebox(0,0)[lb]{\smash{\SetFigFont{10}{12.0}{tt}nonClassOrEnum}}}
\put(2776,-2386){\makebox(0,0)[lb]{\smash{\SetFigFont{10}{12.0}{tt}classOrEnum}}}
\put(4651,-2386){\makebox(0,0)[lb]{\smash{\SetFigFont{10}{12.0}{tt}Symbols ...}}}
\end{picture}

\end{center}
\caption{The structure of the symbol table}
\label{symtabstruct}
\end{BFIGURE}

Before describing the implementation of the support for the
\reserved{namespace} construction 
within the Larch/C++ Checker, some background on the symbol table
system used within the Checker is needed. Figure~\ref{symtabstruct}
illustrates the structure of the system's symbol table at some point.
The basic structure of the symbol table is contained within the
\reserved{class SymTab}. In this illustration, there are two pointers,
\reserved{globalSymTab} and \reserved{currentSymTab}, which point to
objects of type \reserved{SymTab}. \reserved{globalSymTab} always
points to the top of the symbol table, while \reserved{currentSymTab}
always points to the current local scope. Each \reserved{SymTab}
object contains a link to its enclosing scope, \reserved{global}, and
a link to the information in that table's current scope,
\reserved{locals}. At the top level of the symbol table, the
\reserved{globals} pointer is
\reserved{NULL} to signify the top of the table. The local information
in a \reserved{SymTab} is contained with a \reserved{Locale} object. A
\reserved{Locale} contains references to two lists of \reserved{Symbols}, those
that are classes or enumerations, \reserved{classOrEnum}, and those
which are not, \reserved{nonClassOrEnum}. The \reserved{Symbol} class,
and its derived classes, are used to represent and store the various
types of C++ declarations. Please see Figure~\ref{symbolh} for the class
hierarchy diagram for \reserved{Symbol} and its derived classes.

\begin{BFIGURE}
\begin{center}
\setlength{\unitlength}{0.00083300in}%
%
\begingroup\makeatletter\ifx\SetFigFont\undefined
% extract first six characters in \fmtname
\def\x#1#2#3#4#5#6#7\relax{\def\x{#1#2#3#4#5#6}}%
\expandafter\x\fmtname xxxxxx\relax \def\y{splain}%
\ifx\x\y   % LaTeX or SliTeX?
\gdef\SetFigFont#1#2#3{%
  \ifnum #1<17\tiny\else \ifnum #1<20\small\else
  \ifnum #1<24\normalsize\else \ifnum #1<29\large\else
  \ifnum #1<34\Large\else \ifnum #1<41\LARGE\else
     \huge\fi\fi\fi\fi\fi\fi
  \csname #3\endcsname}%
\else
\gdef\SetFigFont#1#2#3{\begingroup
  \count@#1\relax \ifnum 25<\count@\count@25\fi
  \def\x{\endgroup\@setsize\SetFigFont{#2pt}}%
  \expandafter\x
    \csname \romannumeral\the\count@ pt\expandafter\endcsname
    \csname @\romannumeral\the\count@ pt\endcsname
  \csname #3\endcsname}%
\fi
\fi\endgroup
\begin{picture}(6500,5499)(364,-6823)
\put(5116,-1603){\makebox(0,0)[lb]{\smash{\SetFigFont{11}{13.2}{sf}TemplateClassName}}}
\thicklines
\put(3151,-1561){\makebox(6.6667,10.0000){\SetFigFont{10}{12}{rm}.}}
\multiput(2446,-4541)(6.04545,-6.04545){23}{\makebox(6.6667,10.0000){\SetFigFont{7}{8.4}{rm}.}}
\put(2579,-4674){\line( 0, 1){267}}
\multiput(2579,-4407)(-6.06818,-6.06818){23}{\makebox(6.6667,10.0000){\SetFigFont{7}{8.4}{rm}.}}
\put(1644,-4741){\framebox(802,467){}}
\put(1644,-3940){\framebox(1403,467){}}
\put(1644,-3139){\framebox(1937,468){}}
\put(376,-3873){\framebox(801,534){}}
\put(2646,-6611){\line( 1, 0){334}}
\put(2980,-6811){\framebox(1535,467){}}
\multiput(2512,-6611)(6.06818,-6.06818){23}{\makebox(6.6667,10.0000){\SetFigFont{7}{8.4}{rm}.}}
\put(2646,-6744){\line( 0, 1){267}}
\multiput(2646,-6477)(-6.09091,-6.09091){23}{\makebox(6.6667,10.0000){\SetFigFont{7}{8.4}{rm}.}}
\put(2512,-6611){\line( 0, 1){  0}}
\put(1644,-6811){\framebox(868,467){}}
\put(1644,-2271){\framebox(1069,735){}}
\multiput(2713,-1937)(6.04545,-6.04545){23}{\makebox(6.6667,10.0000){\SetFigFont{7}{8.4}{rm}.}}
\put(2846,-2070){\line( 0, 1){267}}
\multiput(2846,-1803)(-6.06818,-6.06818){23}{\makebox(6.6667,10.0000){\SetFigFont{7}{8.4}{rm}.}}
\put(2713,-1937){\line( 0, 1){  0}}
\put(2980,-1803){\framebox(1736,467){}}
\put(2980,-2404){\framebox(1802,400){}}
\multiput(4716,-1603)(6.06818,-6.06818){23}{\makebox(6.6667,10.0000){\SetFigFont{7}{8.4}{rm}.}}
\put(4849,-1737){\line( 0, 1){267}}
\multiput(4849,-1470)(-6.04545,-6.04545){23}{\makebox(6.6667,10.0000){\SetFigFont{7}{8.4}{rm}.}}
\put(4716,-1603){\line( 0, 1){  0}}
\put(4849,-1603){\line( 1, 0){200}}
\put(3314,-4341){\line( 0,-1){1535}}
\put(3314,-5876){\line( 1, 0){534}}
\put(3314,-5075){\line( 1, 0){534}}
\put(3848,-6143){\framebox(1535,534){}}
\put(3848,-5342){\framebox(1602,534){}}
\put(3848,-3740){\framebox(1535,534){}}
\put(2579,-4541){\line( 1, 0){735}}
\put(3314,-3473){\line( 1, 0){534}}
\put(5049,-1803){\framebox(1803,467){}}
\put(3581,-4541){\framebox(2470,534){}}
\put(3647,-4341){\makebox(0,0)[lb]{\smash{\SetFigFont{11}{13.2}{sf}TypedefNonClassOrEnumName}}}
\put(3314,-4274){\line( 1, 0){267}}
\put(3301,-4411){\line( 0, 1){975}}
\multiput(1201,-3586)(6.00000,-6.00000){26}{\makebox(6.6667,10.0000){\SetFigFont{7}{8.4}{rm}.}}
\put(1351,-3736){\line( 0, 1){300}}
\multiput(1351,-3436)(-6.00000,-6.00000){26}{\makebox(6.6667,10.0000){\SetFigFont{7}{8.4}{rm}.}}
\put(1201,-3586){\line( 0, 1){  0}}
\put(1351,-3436){\line( 0, 1){1575}}
\put(1351,-1861){\line( 1, 0){300}}
\put(1351,-3736){\line( 0,-1){825}}
\put(1351,-4561){\line( 1, 0){300}}
\put(1351,-2911){\line( 1, 0){300}}
\put(1351,-3736){\line( 1, 0){300}}
\put(2851,-1786){\line( 0, 1){225}}
\put(2851,-1561){\line( 1, 0){150}}
\put(2851,-2086){\line( 0,-1){150}}
\put(2851,-2236){\line( 1, 0){150}}
\put(1351,-4561){\line( 0,-1){2100}}
\put(1351,-6661){\line( 1, 0){300}}
\put(1845,-4541){\makebox(0,0)[lb]{\smash{\SetFigFont{11}{13.2}{sf}Ident}}}
\put(1644,-3740){\makebox(0,0)[lb]{\smash{\SetFigFont{11}{13.2}{sf}OriginalEnumName}}}
\put(1711,-2938){\makebox(0,0)[lb]{\smash{\SetFigFont{11}{13.2}{sf}TemplateNonClassName}}}
\put(443,-3673){\makebox(0,0)[lb]{\smash{\SetFigFont{11}{13.2}{sf}Symbol}}}
\put(1711,-6677){\makebox(0,0)[lb]{\smash{\SetFigFont{11}{13.2}{sf}TraitOp}}}
\put(3113,-6611){\makebox(0,0)[lb]{\smash{\SetFigFont{11}{13.2}{sf}ExtendedTraitOp}}}
\put(1711,-1803){\makebox(0,0)[lb]{\smash{\SetFigFont{11}{13.2}{sf}Original}}}
\put(1711,-2004){\makebox(0,0)[lb]{\smash{\SetFigFont{11}{13.2}{sf}Namespace}}}
\put(1711,-2204){\makebox(0,0)[lb]{\smash{\SetFigFont{11}{13.2}{sf}Name}}}
\put(3113,-1603){\makebox(0,0)[lb]{\smash{\SetFigFont{11}{13.2}{sf}OriginalClassName}}}
\put(3047,-2271){\makebox(0,0)[lb]{\smash{\SetFigFont{11}{13.2}{sf}NamespaceAliasName}}}
\put(4048,-5943){\makebox(0,0)[lb]{\smash{\SetFigFont{11}{13.2}{sf}BuiltInTypeName}}}
\put(3914,-5142){\makebox(0,0)[lb]{\smash{\SetFigFont{11}{13.2}{sf}TypedefEnumName}}}
\put(3914,-3539){\makebox(0,0)[lb]{\smash{\SetFigFont{11}{13.2}{sf}TypedefClassName}}}
\end{picture}

\end{center}
\caption{The \reserved{Symbol} class hierarchy}
\label{symbolh}
\end{BFIGURE}

Modifying the Larch/C++ system to support the \reserved{namespace} and
\reserved{using} constructs was not difficult. Within the Larch/C++
checker, there was already support for the parsing of the
\reserved{namespace} and \reserved{using} syntax. \reserved{Symbol}s of
the form
\reserved{OriginalNamespaceName} were created when a
\reserved{namespace}  declaration was parsed. The
\reserved{OriginalNamespaceName} object contains a local symbol table
with a structure similar to the system's symbol table. The declarations
within a namespace are placed within this local table so that they are
available whenever the namespace object is referenced. Since this
portion of the implementation was already complete, all that needed to
be done was to implement the semantics of the using declarations and
directives.

At first, the best solution was thought to be that when a \reserved{using}
directive or declaration was seen, simply copy the symbol for that
declaration out of the namespace's local symbol table into the current
symbol table for that scope. However, the question was raised, are the
symbols we copy really like the other symbols within the table? The
declarations introduced by \reserved{using} declarations and directives appear to
be or are synonyms for the previous declarations. It was decided that
the declarations introduced by the \reserved{using} forms should be
differentiated from other forms of declarations. This was done because 
\reserved{using} directives and declarations don't really create new declaration
information, they simply change its visibility. Thus simply creating
new regular declaration objects that were copies of the previous
declaration objects would not suffice.

Various solutions to this problem were proposed. One solution involved
the creation of a new form of \reserved{Symbol}, the \reserved{Alias}
which would have contained a reference back to the original symbol in
the namespace symbol object. These \reserved{Alias} objects would have been
placed into the scope containing the \reserved{using} form. While this
solution had promise, it also added complexity. The actual declaration
information would not have been easily accessible within the \reserved{Alias}
object because it would have had to have been dereferenced in some manner.

While the \reserved{Alias} object idea was rejected, the basic idea
behind it, the idea of aliases, formed the cornerstone for the actual
solution. The basic structure of the \reserved{Symbol} class and its
derived classes was modified to support a new data member, the alias
field. The \reserved{alias} field is a boolean value that stated whether this
object was an alias or a simple symbol. Besides the data member,
member functions for the observation and modification of this data
member were created. The use of C++ and inheritance shined at this
point. Only the code for the \reserved{Symbol} class had to be
modified, with the change affecting the derived classes via
inheritance. This solution also avoided the extra complexity of creating a
new type of \reserved{Symbol} and the difficulty in getting at the declaration
information. When a \reserved{using} statement was seen, all that needed to be
done was create a copy of the symbols within the namespace
object, flag these copies as aliases, and insert them into the symbol
table.

The code to create the alias copies and do the insertion into the
symbol table ended up going through two revisions. The first
implementation took a non object-oriented path to the
solution. Remember that the \reserved{namespace} objects contained a
symbol table with the same structure as the global symbol table
described earlier. To create copies of the symbols, the system needed
to be able to access them easily. The first implementation modified
the
\reserved{Locale} class, adding member functions that gave access to
the protected data members. Once the system had the lists of
\reserved{Symbol}s out of the \reserved{Locale}, it processed these
Symbols one at a time via a large case statement. The case statement
checked to see what kind of \reserved{Symbol} it was, created a copy
of the symbol with the \reserved{alias} field set, and passed this
copy back to the system to add to the current scope. While this
solution worked, it seemed to void some of the advantages of
object-oriented design. First, member functions should not pass
private or protected data members to their callers. This breaks the
property of encapsulation. Second, with the use of virtual functions,
there should be no need for large case statements to differentiate
between the type of symbols. Thus, the code was rewritten to be more
object-oriented.

The final implementation consists of the following. The \reserved{Symbol} class
and its derived classes were modified to have member functions
\reserved{CloneSym} and \reserved{CloneSymAsAlias}. A \reserved{Symbol}
object receiving one of these method calls creates a copy of itself,
setting the \reserved{alias} flag as necessary. The functions to allow
for access to the data within the \reserved{Locale} objects and the
\reserved{SymTab} objects were moved into member functions for those
classes. This kept the encapsulation of the protected data members
intact. Figure ~\ref{cpsymcpp} illustrates the
\reserved{CopySymbols} functions and Figure ~\ref{usdlgram} shows
their use in the Larch/C++ grammar file for the case of a
\reserved{using} directive.
\begin{BFIGURE}
\begin{verbatim}

void SymTab::CopySymbols(SymTab *Table, bool asAlias){ 
    locals->CopySymbols(Table,asAlias);
}

void Locale::CopySymbols(SymTab *Table, bool asAlias){
  Table->CopySymbols(asAlias,nonClassOrEnum);
  Table->CopySymbols(asAlias,classOrEnum);
}

void SymTab::CopySymbols(bool asAlias, SymNodePtr SymList){
 
  SymNodePtr currentSym;
  Symbol *copySym;
  currentSym = SymList;

  while (currentSym != NULL) {
    if (asAlias) {
      copySym = Car(currentSym)->CloneSymAsAlias();
    } else {
      copySym = Car(currentSym)->CloneSym();
    }
    if ((this->Defined(Car(currentSym)->GetName()))) {
      Warning(*(currentSym->value->GetName()) +
              DyString(" is multiply defined in this scope\n")+
              DyString("The new definition was ignored and not inserted"));
    } else {
      this->AddSym(copySym);
    }
    SymNodePtr currentSym2 = Cdr(currentSym);
    currentSym = currentSym2;
  }

}
\end{verbatim}
\caption{The \reserved{CopySymbols} functions}
\label{cpsymcpp}
\end{BFIGURE}

\begin{BFIGURE}
\begin{verbatim}
using_directive
    : USING NAMESPACE complete_namespace_name T_SEMI
        @{ @m using_directive.sym;
           @using_directive.sym@ = @complete_namespace_name.sym@;
           SymTab *LocalSyms = @using_directive.sym@->GetLocals();
           /*
             LocalSyms contains a SymTab that contains the
             symbols that must be inserted.
           */
           LocalSyms->CopySymbols(currentSymTab,true);

        @}
\end{verbatim}
\caption{One production for \reserved{using} directives}
\label{usdlgram}
\end{BFIGURE}
The way the functions work for this example is as follows. Looking at
Figure~\ref{usdlgram}, notice the comment. The \reserved{LocalSyms}
variable contains the local symbol table associated with the
namespace. The function \reserved{CopySymbols} is overloaded for
\reserved{SymTab} objects. In this case the call
\begin{verbatim}
LocalSyms->CopySymbols(currentSymTab,true)
\end{verbatim}
 resolves to the first function in Figure~\ref{cpsymcpp}. The arguments
to this function are the destination symbol table into which the
copies will be placed, and a boolean value representing whether the
copies should be aliases. This function then calls
\reserved{Locale::CopySymbols} passing along the
arguments. The \reserved{Locale::CopySymbols} passes the work onto the
more complex of the two \reserved{SymTab::CopySymbols} functions to do
the actual copying of the symbols and insertion within the symbol
table. The insertions are done as follows. The first item is taken off
of the list of symbols. The item is copied either as an alias or as a
plain
\reserved{Symbol} object depending on the value of the
\reserved{asAlias} parameter. The 
copy is then inserted into the symbol table referenced by the default
parameter if it is not already there. The next Symbol to copy is taken
off the list, and the process continues until all symbols have been
copied.

\subsection{Support for LSL Constructs}
As its name suggests, Larch/C++ is dependent upon the Larch Shared
Language. Specifiers use LSL traits to create the abstract models for
their specifications. These models can then be referenced in Larch/C++
in a number of ways: \reserved{uses} sequences, \reserved{expects}
sequences, etc. The Larch/C++ Checker needs to be able to extract
information about the operations defined within traits. The system
must handle the overloading of trait operations and the renaming of
sort parameters within traits. One possible solution was to write a
filter for LSL traits that would extract the necessary
information. This appeared to be a time sink, however. It was decided
to see how other languages in the Larch family handled this interface.

\subsubsection{Interface to the LSL Checker}
\label{lslinterf}
After examining an early implementation for the LCLint
tool~\cite{lclint}, it was realized that the LSL Checker handles the
renaming of traits in
\reserved{includes} clauses, and will generate the needed information
about the trait operations via the \reserved{-syms} option. Reusing
this functionality would save time and effort. Thus, an interface to
the LSL Checker was needed. The interface and its development occurred
in a series of steps. First, the information within the Larch/C++
specifications needed to be converted into a form useful to the LSL
Checker. Then a system of passing this information to the LSL Checker
and capturing the output was needed. Finally the information from the
LSL Checker needed to be translated and stored in a form that the
Larch/C++ Checker could later use.

The first step, converting Larch/C++ \reserved{uses} clauses into useful LSL
information, was a non-trivial task. For example, there needed to be a
way to convert a \reserved{uses} clause like 
\begin{verbatim}
//@ uses SimpleSetTrait(Elem for E, Set<Elem> for C);
\end{verbatim}
into an equivalent LSL \reserved{includes} line. To do this, the
Larch/C++ parser was modified to build strings of trait
information. Two attributes were added to the parse tree nodes, a
pointer to a
\reserved{DyString} called \reserved{info} and a boolean called
\reserved{TooComplex}. The \reserved{info} attribute points to the
string that will eventually contain the trait and renaming information
for the LSL \reserved{includes} line. The string is built as expected,
bottom up, as the parse is conducted. The system handles template
types, like the type \reserved{Set<int>} above, by translating the
\reserved{<>} symbols into legal \reserved{[]} symbols. The system can
do this because Larch/C++ requires that template types have equivalent
sorts~\cite[Page 20]{Leavens96c}.However, C++ allows for complex
expressions as template arguments. How should the system handle those?
The answer is, it does not. The \reserved{TooComplex} attribute is
used to keep track of the simplicity of the template arguments. If
something that the system cannot handle, such as an expression
\reserved{1 < 2}, appears as an argument, the \reserved{TooComplex}
flag is set. This causes the Checker to generate a warning that the
\reserved{uses} clause was too complex to translate into an LSL
\reserved{includes} directive.

Once the information in a \reserved{uses} clause has been translated,
it needs to be passed to the LSL Checker. This requires some sort of
interface between the Larch/C++ Checker and the LSL Checker. The
interface used owes a lot of its basic design to the interface used by
LCLint 1.4a~\cite{lclint}. The LCLint tool generates a temporary trait
file that contains an \reserved{includes} statement for the trait
information. It then uses the Standard C \reserved{system} call to run
the LSL Checker over that file, redirecting the Checker's output into another
temporary file. If \reserved{system} returns an error condition, then the LSL
Checker is run again over the file to allow the error messages to be
passed to the user.

As in the LCLint tool, the Larch/C++ Checker takes the
information generated from the \reserved{uses} clause (or other expressions
involving traits) and creates a temporary trait. This trait consists
of an
\reserved{includes} statement that uses the generated traits. The LSL
Checker is then run on this temporary trait to both check its syntax and to
generate the \reserved{-syms} output. Once this basic interface 
was sketched out, a few questions surfaced. How are the temporary
files generated and managed? What is the best way to call the LSL
Checker? Does the LSL Checker need to be run twice?

Temporary file management was the first item to be addressed. It was
known that there was a \reserved{tmpfile} function in Standard C that
would generate a uniquely\-named temporary file. The problems with this method
were that the file was always automatically deleted, and that the
return value of the call was a \reserved{FILE *}. Since the system was
implemented in C++, it was felt that the system should use streams as
the interface rather than \reserved{FILE} pointers. The system should
also have the ability to keep its temporary files around both for
debugging purposes, and perhaps for caching purposes in the future
(see Section ~\ref{futwork}). It was also necessary to design the
implementation of temporary files in a way that would be more easily
portable to other operating systems.

The best solution to all of these issues was to design a temporary
file class. The class would encapsulate the creation of the file names
within its constructor, allowing us the ability to provide our own
name or let the system choose one. It would also provide the ability
to flag a file as undeleteable, so that it would not be reclaimed by
the system. The destructor could check this flag, and delete the file
as necessary. Figure~\ref{tmpfile} shows the header for the class
\reserved{TmpFile} designed to meet these goals.

\begin{BFIGURE}
\begin{verbatim}
#ifndef _TmpFile_h
#define _TmpFile_h 1
#include <fstream.h>
#include <stdio.h>
#include <stdlib.h>
#include "DyString.h"
#include "dirname.h"

class TmpFile {

public:
  TmpFile();
  TmpFile(char *extension);
  virtual ~TmpFile();
  virtual bool open();
  virtual bool close();
  virtual char *GetPath() const;
  virtual char *GetFileName() const;
  virtual char *GetCompletePath() const;
  virtual bool IsDeleteable() const;
  virtual void Deleteable(bool flag);
  friend ostream& operator << (ostream& out, const TmpFile& tmp);
  static void SetDefaultPath(char *def_dir);

protected:
  static DyString default_path;
  DyString path;
  DyString filename;
  DyString extension;
  DyString CompletePath;
  bool deleteable;
  fstream mystream;
};
#endif
\end{verbatim}
\caption{The \reserved{TmpFile} header}
\label{tmpfile}
\end{BFIGURE}
\reserved{TmpFile}, which is part of the common code for the system, is an
abstract class. Derived classes may be specialized to allow for
different behaviors. In the Larch/C++ Checker, a derived class
\reserved{OutTmpFile} generates temporary files for system outputs. The classes
carry along their path information, which is based upon a default path
stored in a static member variable. The value of this variable defaults to
the current directory, but may be reset as needed. The \reserved{Deleteable}
member function is used to change the deleteable flag on the file. If
the deleteable flag is true, the file will be deleted when its
destructor is called. The class currently allows for an extension to
be added to the temporary file. This was added because LSL requires
traits to reside in files with the extension
\reserved{.lsl}.

Once a system for generating temporary files was ready, it was time to
design the method for calling the LSL Checker. All of the efforts
surrounding the call are encapsulated within the \reserved{call\_lsl}
function. Figures~\ref{calllsl1} and~\ref{calllsl2} show the source
code for this function. The first half of the function
(Figure~\ref{calllsl1}) sets up the temporary files that are needed. It creates
two \reserved{OutTmpFile}s, \reserved{tmptrait} to hold the temporary trait information, and
\reserved{symsout} to hold the output from the LSL Checker. The function checks
to make sure that these files can be opened for writing. If not, it
exits with an error. Once it has opened the files, it proceeds to
create the temporary trait. Traits in LSL are required to have the
trait name and the file name be the same. Thus the first thing
\reserved{call\_lsl} does is get the file name from
\reserved{tmptrait}. It takes the name provided by the
\reserved{OutTmpFile} and the \reserved{includes} line 
passed as an argument, and creates the temporary trait. For example,
given the \reserved{uses} clause
\begin{verbatim}
//@ uses SimpleSetTrait(Elem for E, Set<Elem> for C);
\end{verbatim}

\reserved{call\_lsl} will be called with the \reserved{includes}
parameter set to
\begin{verbatim}
SimpleSetTrait(Elem for E, Set[Elem] for C)
\end{verbatim}
and will generate a temporary trait similar to the following.
\begin{verbatim}
eaaa04712:trait
includes SimpleSetTrait(Elem for E,Set[Elem] for C)
\end{verbatim}

\begin{BFIGURE}
\begin{verbatim}
#include <signal.h>
#include <stdlib.h>
#include "OutTmpFile.h"
#include "LSLsup.h"
#include "debug.h"
#include "DGetEnv.h"
#include "execit.h"

extern int lslparse();
extern bool LSL_keep_on_error;
extern bool verbose;

int call_lsl(DyString includes){
  extern bool debug_flag;
  int val,pid,status;
  DyString sys_str;
  
  OutTmpFile tmptrait("lsl");
  OutTmpFile symsout;
  if(debug_flag){
    tmptrait.Deleteable(false);
    symsout.Deleteable(false);
  }
  if(!tmptrait.open()){
    return(TMPFILE_ERR);
  }
   if(!symsout.open()){
    return(TMPFILE_ERR);
  }
  tmptrait.write(tmptrait.GetFileName());
  tmptrait.write(":trait\n");
  tmptrait.write("includes ");
  tmptrait.write(includes.ToCppString());
  tmptrait.write("\n");
  tmptrait.close();
\end{verbatim}
\caption{The \reserved{call\_lsl} function, part one}
\label{calllsl1}
\end{BFIGURE}

The second part of \reserved{call\_lsl} handles the generation of the
call to the LSL Checker and any resulting errors from that call; then it
passes the results on to the next stage. \reserved{call\_lsl} uses 
the environment variable \reserved{LSL\_EXE\_PATH} to locate the LSL
Checker in the operating system. The use of an environment variable here was
more flexible than either using a compilation macro to set the path,
or simply choosing a path. The environment variable makes it easy for
users to customize their installation of the tools. A call to the
\reserved{system} function is built from the path to the LSL
Checker and  the name of the temporary files. The macro variable
\reserved{EXECIT} is used to prepend an \reserved{exec} to the front of the
string on systems that support \reserved{exec}. This was done to improve the
ability of the system to handle user interrupts. The first version of
\reserved{call\_lsl} did not use \reserved{exec} and it had a tendency to not
report errors or respond to user signals. For example, if a user hit
Ctrl-C in the middle of checking, the system would return the user to
the command prompt, but continue to execute. The abnormal behavior was
traced to the shell invoked by the \reserved{system} call catching the signals
that should have been passed upward. The addition of the \reserved{exec} to
the command line passed to the \reserved{system} call allows for the messages
to return back to \reserved{system} as expected. 

After the actual \reserved{system} call, most of the remainder of
\reserved{call\_lsl} handles error conditions that may be
returned. The return value of \reserved{call\_lsl} is checked within
the parser. If the system has been interrupted, this fact is passed up
the call chain to the system driver, which halts the system. If the
\reserved{system} call has executed correctly, the Larch/C++ Checker
is ready to move on to parse the resulting \reserved{-syms} output.

\begin{BFIGURE}
\begin{verbatim}
  if(verbose){
    cerr << "Checking traits: " << includes.ToCppString() << endl;
  }
  DyString * lsl_exe = DGetEnv("LSL_EXE_PATH", new DyString("lsl"));
  sys_str = DyString(EXECIT) + *lsl_exe + DyString(" -syms ") +
            DyString(tmptrait.GetFileName());
  sys_str = sys_str + DyString(" > ") + DyString(symsout.GetFileName());
  val = system(sys_str.ToCppString());
  if(val < 0){
    return(SYSCALL_ERR);
  }
  if (WIFEXITED(val) && WEXITSTATUS(val) != 0) {
    if (LSL_keep_on_error) {
      tmptrait.Deleteable(false);
    }
    return(LSL_ERR);
  }
  if (WIFSIGNALED(val)) {
    tmptrait.close();
    return(WTERMSIG(val));
  }
  symsout.close();
  tmptrait.close();
  val = parse_lsl(symsout.GetFileName());
  return(val);
}
\end{verbatim}
\caption{The \reserved{call\_lsl} function, part two}
\label{calllsl2}
\end{BFIGURE}

Once the basic interface was present in \reserved{call\_lsl}, and
before the parser for the
\reserved{-syms} output was written, the
interface to the LSL Checker was tested. At first everything seemed to
work fine. However, on very large traits the Larch/C++ checker consistently
caused core dumps. After investigation, it was discovered that the LSL
Checker had a large memory footprint that seemed to be causing the
system to run out of memory. After consulting with the maintainers of
the LSL Checker for ideas, the LSL Checker was modified to use the
garbage collector also. This reduced its memory footprint to one-half
the previous size, and allowed the Larch/C++ Checker to execute as
expected. 

Another discovery was that the LSL Checker did not follow
the UNIX convention of sending output that reported errors to the
\reserved{stderr} stream. This made clear why the implementors of the
LCLint tool made two calls to the LSL Checker. Since the errors were
reported on \reserved{stdout}, and they captured \reserved{stdout} to
a file, the second execution was to allow for the errors to appear to
the screen. Since modifications had already been made to support the
garbage collector, the LSL Checker's output system was modified to
support standard UNIX conventions. This allows the Larch/C++ checker
to make a single system call and still get user messages output to
the screen, and the \reserved{-syms} output to a file.

\subsubsection{Data Structures for Sorts}
Before one can parse and store the output from the LSL Checker, there
must be data structures to store it in. The design of the data
structures for sorts needed to embody the concept of sorts as they are
used in LSL, while also being easy to build. The belief was that the
sorts for C++ declarations might need to be built in pieces,
bottom-up, as the parse occurred. This influenced the functionality of
the design.

Originally, the Larch/C++ Checker had one class, \reserved{LCPPSort},
which was used to store sort information. It simply stored the
\reserved{TypeSpecifier} and \reserved{Declarator} information for a
declaration. The design of the new sort system began with the
\reserved{LCPPSort} class so that it would be easy to plug into the
existing tools.

After examining documentation for LSL and a multitude of traits, a
basic idea of what was needed to represent sorts was built. There are
three types of sorts within LSL, atomic sorts, parameterized sorts,
and arrow sorts. \emph{Atomic} sorts are the basic sort building
blocks. An example would be the sort \reserved{int}. Atomic sorts take
no arguments, and cannot be broken down into smaller
pieces. 


\emph{Parameterized} sorts are built by combining atomic sorts, or
other parameterized sorts, into a more complex whole. An example would
be the sort
\reserved{Set[int]}, which consists of the two atomic sorts,
\reserved{Set} and \reserved{int}. Even though most observed
parameterized sorts only had one parameter, the decision was made to
support the general case of an indeterminant number of parameters for a
given sort.

\emph{Arrow} sorts represent the signatures for LSL operators. An
example would be \reserved{int , int -> int}. Arrow sorts consist of
two pieces, an argument list which contains a list of atomic and
parameterized sorts, and a result sort which is either a atomic or
parameterized sort. Note that LSL does not support the concept of
higher order operations; that is, arrow sorts cannot be used as either
arguments or results. This fact simplifies the concrete implementation
of these sorts.

\begin{BFIGURE}
\begin{center}
\setlength{\unitlength}{0.00083300in}%
%
\begingroup\makeatletter\ifx\SetFigFont\undefined
% extract first six characters in \fmtname
\def\x#1#2#3#4#5#6#7\relax{\def\x{#1#2#3#4#5#6}}%
\expandafter\x\fmtname xxxxxx\relax \def\y{splain}%
\ifx\x\y   % LaTeX or SliTeX?
\gdef\SetFigFont#1#2#3{%
  \ifnum #1<17\tiny\else \ifnum #1<20\small\else
  \ifnum #1<24\normalsize\else \ifnum #1<29\large\else
  \ifnum #1<34\Large\else \ifnum #1<41\LARGE\else
     \huge\fi\fi\fi\fi\fi\fi
  \csname #3\endcsname}%
\else
\gdef\SetFigFont#1#2#3{\begingroup
  \count@#1\relax \ifnum 25<\count@\count@25\fi
  \def\x{\endgroup\@setsize\SetFigFont{#2pt}}%
  \expandafter\x
    \csname \romannumeral\the\count@ pt\expandafter\endcsname
    \csname @\romannumeral\the\count@ pt\endcsname
  \csname #3\endcsname}%
\fi
\fi\endgroup
\begin{picture}(5499,5724)(1789,-5773)
\thicklines
\put(4276,-961){\framebox(1875,900){}}
\put(4876,-1861){\line( 1, 0){ 75}}
\put(4951,-1861){\line( 1, 1){300}}
\put(5251,-1561){\line( 1,-1){300}}
\put(5551,-1861){\line(-1, 0){600}}
\put(5551,-1861){\line( 1, 0){150}}
\put(5701,-1861){\line(-1, 0){150}}
\put(3151,-3361){\framebox(1800,900){}}
\put(3751,-4261){\line( 1, 1){300}}
\put(4051,-3961){\line( 1,-1){300}}
\put(4351,-4261){\line(-1, 0){600}}
\put(3751,-4261){\line(-1, 0){1200}}
\put(4051,-3436){\line( 0,-1){525}}
\put(1801,-5761){\framebox(1650,900){}}
\put(5251,-961){\line( 0,-1){600}}
\put(5476,-3361){\framebox(1800,900){}}
\put(5926,-2986){\makebox(0,0)[lb]{\smash{\SetFigFont{12}{14.4}{rm}ArrowSort}}}
\put(4876,-1861){\line(-1, 0){825}}
\put(4051,-1861){\line( 0,-1){600}}
\put(5551,-1861){\line( 1, 0){750}}
\put(6301,-1861){\line( 0,-1){600}}
\put(2551,-4261){\line( 0,-1){600}}
\put(4876,-586){\makebox(0,0)[lb]{\smash{\SetFigFont{12}{14.4}{rm}LCPPSort}}}
\put(2176,-5386){\makebox(0,0)[lb]{\smash{\SetFigFont{12}{14.4}{rm}ParamSort}}}
\put(3601,-2986){\makebox(0,0)[lb]{\smash{\SetFigFont{12}{14.4}{rm}AtomicSort}}}
\end{picture}

\end{center}
\caption{Class hierarchy for \reserved{LCPPSort}: first design}
\label{classfirst}
\end{BFIGURE}

Originally, the class hierarchy in Figure~\ref{classfirst} was
proposed. Within this hierarchy, \reserved{LCPPSort} was viewed as an abstract
class, allowing for basic behavior to be shared, and so that the
implementation could use \reserved{LCPPSort} pointers to hold any type of
sort. The class structure represents the feeling that atomic sorts are
basic and parameterized sorts build upon them. The arrow sorts exist
in their own tree, but contain both atomic and parameterized
sorts. While this design captured the behavior of the various sorts
well, it was uncertain whether it would be practical in practice. As
mentioned earlier, it was believed that the sorts for a C++
declaration might have to be built bottom-up. For example, the sort
for a declaration \reserved{int i;} would first build an atomic sort
\reserved{int}, later converting it to the parameterized sort
\reserved{Obj[int]}. The question was raised whether a design for the
classes could be found where these conversions would not have to
happen. It was also noted that the list of parameters inside of a
parameterized sort and the list of arguments within an arrow sort could be a
mix of atomic and parameterized sorts. This could lead to 
later headaches when trying to decide if two parameterized sorts were
equal. The lack of homogeneity would make the task more complex.

The realization that atomic sorts were equivalent to parameterized
sorts with zero parameters simplified the class hierarchy for sorts to
that shown in Figure~\ref{classfinal}. This hierarchy still supports
the three forms of sorts, while not needing to be able to convert
between atomic and parameterized sorts. With the new hierarchy a
declaration \reserved{int i;} will first build a parameterized sort
\reserved{int} which has no parameters. Later it would create the
final sort \reserved{Obj[int]} by adding the \reserved{int} as a
parameter to the \reserved{Obj} sort. The change in design also made
lists of parameters in both parameterized and arrow sorts homogeneous.

\begin{BFIGURE}
\begin{center}
\setlength{\unitlength}{0.00083300in}%
%
\begingroup\makeatletter\ifx\SetFigFont\undefined
% extract first six characters in \fmtname
\def\x#1#2#3#4#5#6#7\relax{\def\x{#1#2#3#4#5#6}}%
\expandafter\x\fmtname xxxxxx\relax \def\y{splain}%
\ifx\x\y   % LaTeX or SliTeX?
\gdef\SetFigFont#1#2#3{%
  \ifnum #1<17\tiny\else \ifnum #1<20\small\else
  \ifnum #1<24\normalsize\else \ifnum #1<29\large\else
  \ifnum #1<34\Large\else \ifnum #1<41\LARGE\else
     \huge\fi\fi\fi\fi\fi\fi
  \csname #3\endcsname}%
\else
\gdef\SetFigFont#1#2#3{\begingroup
  \count@#1\relax \ifnum 25<\count@\count@25\fi
  \def\x{\endgroup\@setsize\SetFigFont{#2pt}}%
  \expandafter\x
    \csname \romannumeral\the\count@ pt\expandafter\endcsname
    \csname @\romannumeral\the\count@ pt\endcsname
  \csname #3\endcsname}%
\fi
\fi\endgroup
\begin{picture}(5424,3324)(1189,-4273)
\thicklines
\put(3601,-2761){\line( 1, 1){300}}
\put(3901,-2461){\line( 1,-1){300}}
\put(4201,-2761){\line(-1, 0){600}}
\put(3601,-2761){\line(-1, 0){1500}}
\put(2101,-2761){\line( 0,-1){600}}
\put(4276,-2761){\line( 1, 0){1425}}
\put(5701,-2761){\line( 0,-1){600}}
\put(4201,-2761){\line( 1, 0){ 75}}
\put(3901,-1861){\line( 0,-1){600}}
\put(1201,-4261){\framebox(1800,900){}}
\put(5251,-3886){\makebox(0,0)[lb]{\smash{\SetFigFont{12}{14.4}{rm}ArrowSort}}}
\put(4801,-4261){\framebox(1800,900){}}
\put(3001,-1861){\framebox(1875,900){}}
\put(3526,-1486){\makebox(0,0)[lb]{\smash{\SetFigFont{12}{14.4}{rm}LCPPSort}}}
\put(1651,-3886){\makebox(0,0)[lb]{\smash{\SetFigFont{12}{14.4}{rm}ParamSort}}}
\end{picture}

\end{center}
\caption{Class hierarchy for \reserved{LCPPSort}: final design}
\label{classfinal}
\end{BFIGURE}

The final piece of the puzzle for representing sorts was how to
support the overloading of LSL operators. Remember from earlier
discussion (Section~\ref{lslts}) that LSL allows for a given operator
name to have many different signatures associated with it. The choice
of which signature to use is then made based upon the context of the
use of the operator. The data structure needed to handle basic insertion,
and insertion of duplicates easily. It would be convenient if it could
handle retrieval based on different pieces of the arrow sort
structure. The design chosen was an iterated set of arrow sorts. This
data structure had the advantage of easily handling insertion,
especially insertion of duplicates. Although the data structure itself
does not handle retrieval of a given arrow sort by the parts of arrow
sorts, i.e. by either parameter list or result sort, helper functions
could provide that functionality.

Figure~\ref{ArrowSet} shows the class \reserved{ArrowSet} used to
implement the set of arrow sorts. The set is actually built on top of
a singly linked list provided by the macros in the
\reserved{SINGLYLL.h} file. The functionality is as expected for
sets with the member functions modeling the usual mathematical
behavior for sets. The member functions \reserved{Union},
\reserved{Intersection}, and \reserved{Diff} generate their results
within the default parameter, rather than generating a completely new
set. For example, given sets \reserved{S1} and \reserved{S2}, the
operation \reserved{S1.Union(S2)} will change \reserved{S1} to hold
\reserved{S1} $\cup$ \reserved{S2}. This behavior, while documented in
the specification for the class, can be somewhat confusing.

\begin{BFIGURE}
\begin{verbatim}
#include "SINGLYLL.h"
#include "ArrowSrt.h"
typedef ArrowSort *ArrowSortPtr;
DECLARE_SINGLYLL(ArrowSortPtr,ArrowSortNode)
typedef ArrowSortNode *ArrowSortNodePtr;
class ArrowSet{
public:
  ArrowSet();
  ArrowSet(const ArrowSortPtr item);
  ArrowSet(const ArrowSet& set);
  bool In(const ArrowSortPtr item) const;
  int Cardinality() const;
  bool IsEmpty() const;
  bool Subset(const ArrowSet& set) const;
  bool Insert(const ArrowSortPtr item);
  bool Remove(const ArrowSortPtr item);
  void Union(const ArrowSet& set1);
  void Intersection(const ArrowSet& set1);
  void Diff(const ArrowSet& set1);
  bool operator == (const ArrowSet& set1) const;
  bool operator != (const ArrowSet& set1) const;
  void Save_Iterate();
  void Restore_Iterate();
  void Reset_Iterate();
  ArrowSortPtr Iterate();
protected:
  
  ArrowSortNodePtr theSet;
  ArrowSortNodePtr iteratePoint;
  ArrowSortNodePtr old_iteratePoint;
};
ostream& operator << (ostream& out, ArrowSet item);
\end{verbatim}
\caption{The \reserved{ArrowSet} class}
\label{ArrowSet}
\end{BFIGURE}
The major addition to the implementation of set shown here is the
ability to iterate through the items in the set. Most sets allow you
to look to see if an item is in a set. When dealing with sets of arrow
sorts, however, the user might only know the result sort he is looking
for. This means that there needs to be a way to examine each member in
the set, allowing for comparisons with the individual parts of an
arrow sort. Thus the iterator, and its controlling functions, were
added. The \reserved{Reset\_Iterate} function clears any information
about previous iterations and makes the set ready to iterate. The
\reserved{Iterate} function returns the next member of the set in
the iteration, or NULL if there are no elements remaining. The
\reserved{Save\_Iterate} and \reserved{Store\_Iterate} member
functions were added to help support the ability to print out the
sets. The \reserved{operator <<} function was written using the
\reserved{Iterate} function to output one member at a time. However,
what if a user wanted to print out the set in the middle of an
iteration without losing the iteration
information. \reserved{Save\_Iterate} saves the current iteration
point, which can be later restored via
\reserved{Restore\_Iterate}. The use of all of the iteration
functions is illustrated in the code for \reserved{operator <<} shown
in Figure~\ref{oppto}.
\begin{BFIGURE}
\begin{verbatim}
ostream& operator << (ostream& out, ArrowSet item){
  
  ArrowSortPtr val;
  int iter;

  iter = 0;
  out << "{";
  item.Save_Iterate();
  item.Reset_Iterate();
  val = item.Iterate();
  while(val != NULL){
    if (iter == 0){
      val->Print(out);
      iter++;
    }
    else {
      cout << ",";
      val->Print(out);
    }
    val = item.Iterate();
  }
  out << "}";
  item.Restore_Iterate();
}
\end{verbatim}
\caption{The \reserved{operator <<} function for \reserved{class
  ArrowSet}}
\label{oppto}
\end{BFIGURE}

\subsubsection{The Operator Signature Parser}
Once the LSL Checker has been used to generate the \reserved{-syms}
output, there needs to be a way of getting that information into the
Larch/C++ Checker's symbol table. A tool was needed that would convert
the \reserved{-syms} output into operator names and associated arrow
sorts. The development of this portion of the system again looked to
the previous work of LCLint for an idea of how to proceed. The LCLint
tool used a hand-crafted lexer, combined with a \reserved{yacc}-based
parser to generate data structures from the
\reserved{-syms} data. A close examination revealed that the
hand-crafted lexer was too complex to be easily converted for use
within the Larch/C++ Checker. This led to the development of a
\reserved{flex}-based lexer for this input. Starting from the token
definitions in LCLint's signature file and the information in the LSL
technical report \cite{Guttag-Horning-Modet90}, a set of regular
expressions was developed to represent the tokens. The major
difficulty here was that some characters had multiple meanings
depending upon context. For example, the signature
\begin{verbatim}
[ __ , __ ] : int , int -> Tuple[int]
\end{verbatim} 
represents a function named \reserved{[]} which takes two
\reserved{int} arguments and constructs a \reserved{Tuple[int]}. At
first glance this does not seem difficult to break into
tokens. However, the square brackets that form the name of the
function are actually the tokens \reserved{OPENSYM} and
\reserved{CLOSESYM}, while the square brackets in the sort
\reserved{Tuple[int]} are the tokens \reserved{LBRACKET} and \reserved{RBRACKET}. How should
the system be designed to return the correct tokens? The solution was
to create a lexer with start states to allow for the differentiation
of the uses for square brackets. \emph{Start states} allow \reserved{flex} to use
different matching rules based upon what has been seen in the
input. In this case, the switching of states in the lexer is based
upon having seen either a colon or a new-line character. The lexer
knows that when it has seen a colon, it is within the signature for
the operation. Thus the lexer always returns \reserved{LBRACKET} and
\reserved{RBRACKET} in 
this case. When the lexer reaches a new-line character, it knows that
it has reached the end of a complete signature. At this point it
switches to a start state where square brackets are \reserved{OPENSYM}s and
\reserved{CLOSESYM}s. Please see
\cite{Levine-Mason-Brown92} for more details on the use of start states.

The initial testing of the operator signature parser revealed a minor
problem. The system was consistently returning parse errors for input
that looked correct. The problem turned out to be that the LSL Checker
automatically broke long lines by inserting a
new-line character and continuing the output on the next line. This
was done to allow for users to easily read long lines that appeared on
their screens. However, the insertion of the new-line character was
causing the lexer for the signatures to switch states prematurely,
leading to the parse errors. There were a number of possible
solutions. One possibility was to change the lexer to use a different
set of start states. While this was probably possible, the lexer as
it existed was relatively simple to understand. If the lexer was to be
modified, it would probably become more complex and difficult to
understand. The other solutions were all based upon modifying the LSL
Checker to change its output format. Any change to the output,
however, had to preserve the output format used by the Larch Prover,
and should continue to make the messages readable to the user. Since
all output in the LSL Checker passed through the same output
functions, some sort of conditional check was needed to turn on and
off the output of the offending new-line. An additional boolean
variable, \reserved{PrintingSyms} was added to the system to flag when
the \reserved{-syms} output was being printed. If the flag is true,
then no new-lines are inserted into the output. This allows the parser
to work as expected.

The final piece of the operator signature parser is the interface to
the Larch/C++ symbol table. This involved the creation of a new type
of \reserved{Symbol} to hold the operator's information and a
modification to the \reserved{Locale} portion of the symbol
table. New classes \reserved{TraitOp} and
\reserved{ExtendedTraitOp} were derived from the \reserved{Symbol}
base class. Basically \reserved{TraitOp} objects look
like all other \reserved{Symbol}s, especially \reserved{Ident}s, except for the
fact that their sort information is an 
\reserved{ArrowSet} not a simple \reserved{LCPPSort}. \reserved{ExtendedTraitOp}
is derived from \reserved{TraitOp} and has an extra name field that is
used in reporting errors. For example, the signature \reserved{if \_\_
then \_\_ else \_\_: Bool, Bool, Bool -> Bool} creates an
\reserved{ExtendedTraitOp}. The name used to index the function in the
symbol table is \reserved{ifthenelse}, while the name reported for
errors is \reserved{if \_\_ then \_\_ else \_\_}. Other signatures, such as
\reserved{f: int -> float}, generate \reserved{TraitOps} because the
name reported for errors is the same as the name used to index the
operation in the symbol table. 

The parser creates a local symbol table into which it places
\reserved{TraitOp} and \reserved{ExtendedTraitOp} symbols as it parses
the signatures. It is at this point where the system does the name
conflict resolution described in Section~\ref{lcppts}. If the
parser-generated symbol has a name that already exists in the 
\reserved{currentSymTab}, the trait operation is discarded, and a
warning is reported to the user. Similarly, if the system is inserting a C++
identifier into the symbol table and discovers an existing trait operation
with the same name, it discards the trait operation and warns the user.

The other major modifications to the symbol table involved adding an
additional field to the \reserved{Locale} class, with its associated
member functions, and the modification of the \reserved{SymTab} class
to allow for copying of the new fields. Figure~\ref{symtabstruct}
shows the old structure of the symbol table. \reserved{Locale}s
contained two namespaces, \reserved{nonClassOrEnum} and
\reserved{classOrEnum}. This form was modified to add a third field,
\reserved{TraitOps}, to hold lists of the trait operations. This was
needed to support the split of the symbol table into separate C++ and
LSL operator ``worlds'', as described in Section~\ref{lcppts}. The
addition of this field required the creation of supporting functions
that operated upon this new field. The new functions are simply copies
of the existing \reserved{Locale} functions which operate upon this
new field. All operations on the new field were kept separate to make
sure that the existing behavior of the Larch/C++ Checker did not
change. Of these new functions, the only one examined in detail here
is the new \reserved{CopyTraitOps} series of functions within
\reserved{Locale} and
\reserved{SymTab}. These functions are basically clones of the
\reserved{CopySymbols} functions described in Section~\ref{namspcimp} modified
to work with the new \reserved{TraitOps} field in the
\reserved{Locale} class. These functions are used to copy the local
symbol table created by the LSL operator signature parser into the
Larch/C++ Checker's main symbol table. Please see the implementation
for further details on all of the new functions.

The operator signature parser is used for more than support for
\reserved{uses} clauses within specifications. It is also used to load
all of the built-in traits, such as the traits for the basic C++
types, in the initialization of the Larch/C++ Checker. This
information is stored at the top level of the symbol table, outside of
any C++ scope so that it will not be removed by the name conflict
resolution system.

\subsubsection{Conversion of C++ declarations to LSL sorts}

The final major piece of support required for sort checking within the
Larch/C++ Checker is the ability to convert C++ declarations into
their equivalent sorts. Recall from the discussion of the rules for
state functions, Section~\ref{staterules}, that C++ declarations
consist of an abstract value wrapped within an object. This
interaction between the sorts of the abstract values and the sorts of
the enclosing objects can make it difficult to automatically generate
sorts. Examples of declarations and their equivalent sorts taken from
the Reference Manual ~\cite[Chapter 5 and Section 6.1.8.1]{Leavens96c}
are illustrated in Figures~\ref{globalsorts} and
~\ref{formalsorts}. Notice that the same form of declaration may have
a different sort depending upon whether it is a declaration of a
variable or a declaration of a formal parameter. Notice also the
complexities surrounding the sorts for
\reserved{struct}, \reserved{union}, and \reserved{class} types. These
special cases make the code for the translation more complex.


\begin{BFIGURE}
\begin{verbatim}
Declaration             Sort of x (used as global)
-------------           --------------------------
const int &const x;     ConstObj[int]
const int *const x;     ConstObj[Ptr[ConstObj[int]]]
const int x[3][4];      Arr[Arr[ConstObj[int]]]
int *x[10];             Arr[Obj[Ptr[Obj[int]]]]
struct IntList {
  int val;
  IntList *next;
  };
IntList x;              ConstObj[IntList]
const IntList x;        ConstObj[Const[IntList]]
int (*x)(int i);        Obj[Ptr[ConstObj[cpp_function]]]
int (* const x)(int i); ConstObj[Ptr[ConstObj[cpp_function]]]
int (*x[10])(int i);    Arr[Obj[Ptr[ConstObj[cpp_function]]]]
\end{verbatim}
\caption{Sorts for global C++ declarations}
\label{globalsorts}
\end{BFIGURE}

\begin{BFIGURE}
\begin{verbatim}
Declaration      VarId   Its Sort (when used as a formal parameter)
-------------    -----   -----------------------------------------
int i            i       int
int & ir         ir      Obj[int]
int * ip         ip      Ptr[Obj[int]]
int ai[]         ai      Ptr[Obj[int]]
struct IPair {
  int fst, snd;
  };
Ipair sip;       sip     Val[IPair]
union FI {
  float f;
  int i;
  };
FI fi;           fi      Val[FI]
\end{verbatim}
\caption{Sorts for formal parameter declarations}
\label{formalsorts}
\end{BFIGURE}

At first, the thought was that sort information would have to be
gathered and passed as attributes throughout the parse tree. However,
this method was discarded once the realization occurred that the
parser was already gathering the information necessary for building
sorts within the \reserved{TypeSpecifier} and \reserved{Declarator} objects. In Larch/C++,
as in C++, type specifiers contain a type name and an associated
qualifier (either \reserved{const} or \reserved{volatile}). \reserved{int},
\reserved{unsigned int}, \reserved{enum color}, and \reserved{const
int}, are all examples of valid type
specifiers~\cite{Leavens96c}. Declarators contain the names of the
items being declared along with a list of qualifiers that provide
additional information. Figure~\ref{declquals} lists the qualifiers as
presented in the Larch/C++ Reference Manual \cite{Leavens96c}.
\begin{BFIGURE}
\begin{verbatim}
  Operator     Meaning
  ---------    -----------------
   *            Pointer
   ::*          Pointer to Member
   &            Reference
   []           Array
   ()           Function
\end{verbatim}
\caption{Declarator qualifiers}
\label{declquals}
\end{BFIGURE}


After it was realized that the type specifier and declarator
information was enough to build sorts, the question became, at what
point during a parse is this information complete and available? It
turns out that the key was within the function
\reserved{Declare}. \reserved{Declare} is used to add declarations
to the current symbol table. It was clear at this point that all of
the information required would be available. In examining
the \reserved{Declare} function, it was noted that at certain points
it made calls to the \reserved{LCPPSort} constructor. It was decided that
this would be the best place to generate sorts. 

\begin{BFIGURE}
\begin{verbatim}
ParamSort *BuildSort(const TypeScfr *typescfr, 
                    DQNode *dqlist, bool IsMember){

  ParamSort *typescfrsort, *result;
  ParamSort *tmp1, *tmp2, *tmp3;
  DQNode *currlist, *revlist;

  if(typescfr == NULL){
    result = NULL;
  } else {
    if(typescfr->isConst){
      tmp1 = new ParamSort(DyString("ConstObj"),NULL);
    }
    else {
      tmp1 = new ParamSort(DyString("Obj"),NULL);
    }
    
    if((typescfr->sym->GetKeywordKind()) 
           == TypedefNonClassOrEnumNameTag){
      tmp1 = (ParamSort *)typescfr->sym->GetLCPPSort();
    } else {
      tmp2 = new ParamSort(*(typescfr->sym->GetName()),NULL);
      tmp1->AddParam(tmp2);
    }

    typescfrsort = tmp1;
    result = typescfrsort;
  }
\end{verbatim}
\caption{Function \reserved{BuildSort}, part one}
\label{buildsrtone}
\end{BFIGURE}
Once the location was known, a function needed to be developed to
convert the type specifier and declarator information into the correct
sorts. A preliminary version of this function for global declarations,
called \reserved{BuildSort}, which does not handle classes,
unions, or structs, is pictured in Figures~\ref{buildsrtone} and
~\ref{buildsrttwo}. The portion of the function in
Figure~\ref{buildsrtone} handles the translation of the type specifier
information into an equivalent sort. It first checks to see if a
\reserved{const} specifier is associated with the declaration. If one
is, it creates a parameterized sort \reserved{ConstObj}; otherwise it
creates the sort \reserved{Obj}. Next it examines the \reserved{Symbol}
associated with this type specifier. The Symbol carries the name of
the type or \reserved{typedef}, for which this specifier was
created. If you examine the class hierarchy for \reserved{Symbol} in
Figure~\ref{symbolh}, you will see that one form of identifier is
\reserved{BuiltInTypeName}. Many declarations carry \reserved{Symbol}s of this
sort whose name fields are the built in types. If the \reserved{Symbol} is not a
\reserved{typedef}, the function takes the name from the \reserved{Symbol},
creates a parameterized sort from the name, and then uses that name as
a parameter to the sort created from the \reserved{const}
specifier. If the \reserved{Symbol} associated with this type specifier is a
\reserved{typedef}, it looks inside that symbol to find the correct
sort to use as a parameter to the previously created sort. For
example, this algorithm works as follows on a declaration of the form
\reserved{int i;}.
\begin{itemize}
\item It creates a sort \reserved{Obj} because there is no
\reserved{const} associated with the declaration.
\item Since the symbol carried along with the \reserved{TypeSpecifier}
is a \\
\reserved{BuiltInTypeName}, the system creates another sort
\reserved{int}.
\item The system uses the second sort as a parameter to the first,
creating the correct sort \reserved{Obj[int]}.
\end{itemize}

It is important to note that this portion of the system is dependent
upon the system creating the correct type names for
\reserved{BuiltInTypeName} \reserved{Symbol}s. When preliminary testing began, it
was discovered that type specifiers such as {unsigned int} did not
create the correct names. Functionality was added to the function
\reserved{TypeScfr::Combine} to generate these names correctly. 

\begin{BFIGURE}
\small
\begin{verbatim}
  currlist = dqlist;
  while(currlist != NULL) {
    DeclQual *tmp = Car(currlist);
    if(tmp->IsPointer()){
      if(tmp->IsConst()){
        tmp1 = new ParamSort(DyString("ConstObj"),NULL);
      }
      else {
        tmp1 = new ParamSort(DyString("Obj"),NULL);
      }
      tmp2 = new ParamSort(DyString("Ptr"),NULL);
      tmp2->AddParam(result);
      tmp1->AddParam(tmp2);
      result = tmp1;
    }
    else if (tmp->IsArray()){
      tmp1 = new ParamSort(DyString("Arr"),NULL);
      tmp1->AddParam(result);
      result = tmp1;
    }
    else if (tmp->IsFunction()){
      tmp1 = new ParamSort(DyString("ConstObj"),NULL);
      if (IsMember){
        tmp2 = new ParamSort(DyString("cpp_member_function"),NULL);
      } else {
        tmp2 = new ParamSort(DyString("cpp_function"),NULL);
      }
      tmp1->AddParam(tmp2);
      result = tmp1;
    }
    else {
    }
    currlist = Cdr(currlist);
  }
  return(result);
}
\end{verbatim}
\normalsize
\caption{Function \reserved{BuildSort}, part two}
\label{buildsrttwo}
\end{BFIGURE}

The second half of \reserved{BuildSort} handles any associated
declaration qualifiers. These are handled by building up a sort in
pieces. The individual pieces are not semantically complete sorts, but 
when it finishes, it will have generated a complete sort. \reserved{BuildSort} starts out by checking to see if a pointer,
\reserved{*} has been seen. Declaration qualifiers carry
information about whether they have been modified with a
\reserved{const} keyword also. If the declaration qualifier is a pointer, the system
generates one of the following incomplete sorts, either \reserved{ConstObj[Ptr]} or \reserved{Obj[Ptr]} from
the pointer qualifier depending upon the \reserved{const}ness of the
pointer. Then the sort created from the type
specifier is added to the incomplete sort as a parameter, yielding a
complete sort. As an example,
look what happens to the declaration \reserved{int * const i;}:
\begin{itemize}
\item The type specifier builds a sort \reserved{Obj[int]} as described
above.
\item The first declaration qualifier is a pointer. The system checks
to see if it is modified by a \reserved{const} keyword. Since it is,
it generates the incomplete sort \reserved{ConstObj[Ptr]}.
\item The system combines the sort from the type specifier and the
declaration qualifier to get the final sort \reserved{ConstObj[Ptr[Obj[int]]]}.
\end{itemize}

The other declaration qualifiers work in a similar way, building the
correct sorts for themselves based upon the definitions in the
Reference Manual~\cite{Leavens96c}. As the algorithm cycles through
the list of qualifiers, it builds the sort from the inside out.

Once the basic declaration qualifier code was built, another
abnormality was discovered in testing. Member functions which were
referenced outside of the actual class declaration were not given the
sort \reserved{cpp\_member\_function} as they should have been. The
class \reserved{Point} in Figure~\ref{Pointlcc} is a specification
that illustrates this problem.
\begin{BFIGURE}
\begin{verbatim}
class Point {
public:
  uses informally "pairs of [x,y] values";
  Point();
  int x_val() const;
  int y_val() const;
  void set_x(int xv);
  void set_y(int yv);
};
extern Point::Point();
behavior {
  constructs self;
  ensures self\post = [0,0];
}
extern int Point::x_val() const;
behavior {
  ensures result = self\any.x;
}
extern int Point::y_val() const;
behavior {
  ensures result = self\any.y;
}
extern void Point::set_x(int xv);
behavior {
  modifies self;
  ensures self\post.x = xv /\ (self\pre).y = (self\post).y;
}
extern void Point::set_y(int yv);
behavior {
  modifies self;
  ensures self\post.y = yv /\ (self\pre).x = (self\post).x;
}
\end{verbatim}
\caption{The Point.lcc specification}
\label{Pointlcc}
\end{BFIGURE}

This specification attempts to redeclare the functions outside of the
class declaration. When this happened, the system had no way of
knowing that the functions were actually member functions and that their sorts
were redeclared as \reserved{cpp\_function}. To solve this problem,
the \reserved{Declare} function was modified to check the \reserved{Symbol} it
was declaring to see if it already has a sort. If it does,
\reserved{Declare} keeps that type and does not create a new one. The
code for this is illustrated in Figure~\ref{declarecode}

\begin{BFIGURE}
\begin{verbatim}
if (sym->GetKeywordKind() == OriginalClassNameTag
    || sym->GetKeywordKind() == TemplateClassNameTag) {
  // it's a constructor
  Symbol *ctorSym = new Ident(sym->GetName(), DD_Declared);
  ctorSym->SetLCPPSort(BuildSort(
                currentTypeScfr, dcltor->dcl_quals,IsMember));
  sym->AddCtor(ctorSym);
} else {
  if(sym->GetLCPPSort() == NULL){
    sym->SetLCPPSort(BuildSort(
               currentTypeScfr, dcltor->dcl_quals,IsMember));
  }
  illFormed = AddCheckingDeclarable(currentSymTab, sym) || illFormed;
}
\end{verbatim}
\caption{Code from \reserved{Declare} checking for previous sort}
\label{declarecode}
\end{BFIGURE} 
The code that performs the check
\begin{verbatim}
sym->GetLCPPSort() == NULL
\end{verbatim}
is making sure that the \reserved{Symbol}'s sort pointer has not
been assigned to previously. If this is the case, the system builds a
sort for the \reserved{Symbol} \reserved{sym}.

As mentioned earlier, formal parameters may have sorts that differ from
similar global declarations. A function \reserved{BuildArgSort} which
works in a similar manner as \reserved{BuildSort} handles the
different formulations.

\subsubsection{Additional Work}

To complete the development of the sort checking system for the
Larch/C++ Checker, additional work needs to be done. The lookup
algorithm described in Section~\ref{lcppts} needs to be implemented
and tested. Also, the support functions for dealing with the
\reserved{ArrowSet} objects need to be defined and
implemented. Finally, the main grammar file needs to be modified to
actually perform the sort checking. It is expected that work will
continue on this portion of the project in the future.


